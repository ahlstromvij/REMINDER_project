
\documentclass[11pt,halfline,a4paper,]{ouparticle}

% Packages I think are necessary for basic Rmarkdown functionality
\usepackage{hyperref}
\usepackage{graphicx}
\usepackage{listings}
\usepackage{xcolor}
\usepackage{fancyvrb}
\usepackage{framed}

% Link coloring
\hypersetup{breaklinks=true,
            bookmarks=true,
            pdfauthor={},
            pdftitle={Supporting Information}
            }


% For knitr::kable functionality
\usepackage{booktabs}
\usepackage{longtable}

%% To allow better options for figure placement
%\usepackage{float}

% Packages that are supposedly required by OUP sty file
\usepackage{amssymb, amsmath, geometry, amsfonts, verbatim, endnotes, setspace}

% For code highlighting I think
\DefineVerbatimEnvironment{Highlighting}{Verbatim}{commandchars=\\\{\}}
\definecolor{shadecolor}{RGB}{248,248,248}
\newenvironment{Shaded}{\begin{snugshade}}{\end{snugshade}}
\newcommand{\AlertTok}[1]{\textcolor[rgb]{0.94,0.16,0.16}{#1}}
\newcommand{\AnnotationTok}[1]{\textcolor[rgb]{0.56,0.35,0.01}{\textbf{\textit{#1}}}}
\newcommand{\AttributeTok}[1]{\textcolor[rgb]{0.77,0.63,0.00}{#1}}
\newcommand{\BaseNTok}[1]{\textcolor[rgb]{0.00,0.00,0.81}{#1}}
\newcommand{\BuiltInTok}[1]{#1}
\newcommand{\CharTok}[1]{\textcolor[rgb]{0.31,0.60,0.02}{#1}}
\newcommand{\CommentTok}[1]{\textcolor[rgb]{0.56,0.35,0.01}{\textit{#1}}}
\newcommand{\CommentVarTok}[1]{\textcolor[rgb]{0.56,0.35,0.01}{\textbf{\textit{#1}}}}
\newcommand{\ConstantTok}[1]{\textcolor[rgb]{0.00,0.00,0.00}{#1}}
\newcommand{\ControlFlowTok}[1]{\textcolor[rgb]{0.13,0.29,0.53}{\textbf{#1}}}
\newcommand{\DataTypeTok}[1]{\textcolor[rgb]{0.13,0.29,0.53}{#1}}
\newcommand{\DecValTok}[1]{\textcolor[rgb]{0.00,0.00,0.81}{#1}}
\newcommand{\DocumentationTok}[1]{\textcolor[rgb]{0.56,0.35,0.01}{\textbf{\textit{#1}}}}
\newcommand{\ErrorTok}[1]{\textcolor[rgb]{0.64,0.00,0.00}{\textbf{#1}}}
\newcommand{\ExtensionTok}[1]{#1}
\newcommand{\FloatTok}[1]{\textcolor[rgb]{0.00,0.00,0.81}{#1}}
\newcommand{\FunctionTok}[1]{\textcolor[rgb]{0.00,0.00,0.00}{#1}}
\newcommand{\ImportTok}[1]{#1}
\newcommand{\InformationTok}[1]{\textcolor[rgb]{0.56,0.35,0.01}{\textbf{\textit{#1}}}}
\newcommand{\KeywordTok}[1]{\textcolor[rgb]{0.13,0.29,0.53}{\textbf{#1}}}
\newcommand{\NormalTok}[1]{#1}
\newcommand{\OperatorTok}[1]{\textcolor[rgb]{0.81,0.36,0.00}{\textbf{#1}}}
\newcommand{\OtherTok}[1]{\textcolor[rgb]{0.56,0.35,0.01}{#1}}
\newcommand{\PreprocessorTok}[1]{\textcolor[rgb]{0.56,0.35,0.01}{\textit{#1}}}
\newcommand{\RegionMarkerTok}[1]{#1}
\newcommand{\SpecialCharTok}[1]{\textcolor[rgb]{0.00,0.00,0.00}{#1}}
\newcommand{\SpecialStringTok}[1]{\textcolor[rgb]{0.31,0.60,0.02}{#1}}
\newcommand{\StringTok}[1]{\textcolor[rgb]{0.31,0.60,0.02}{#1}}
\newcommand{\VariableTok}[1]{\textcolor[rgb]{0.00,0.00,0.00}{#1}}
\newcommand{\VerbatimStringTok}[1]{\textcolor[rgb]{0.31,0.60,0.02}{#1}}
\newcommand{\WarningTok}[1]{\textcolor[rgb]{0.56,0.35,0.01}{\textbf{\textit{#1}}}}

% use upquote if available, for straight quotes in verbatim environments
\IfFileExists{upquote.sty}{\usepackage{upquote}}{}

% For making Rmarkdown lists
\providecommand{\tightlist}{%
  \setlength{\itemsep}{0pt}\setlength{\parskip}{0pt}}

% Macros for dealing with affiliations, footnotes, etc.
\makeatletter
\def\Newlabel#1#2#3{\expandafter\gdef\csname #1@#2\endcsname{#3}}

\def\Ref#1#2{\@ifundefined{#1@#2}{???}{\csname #1@#2\endcsname}}

\newcommand*\samethanks[1][\value{footnote}]{\footnotemark[#1]}

\newcommand*\ifcounter[1]{%
  \ifcsname c@#1\endcsname
    \expandafter\@firstoftwo
  \else
    \expandafter\@secondoftwo
  \fi
}

\newcommand*\thanksbycode[1]{%
  \ifcounter{FNCT@#1}
    {\samethanks[\value{FNCT@#1}]}
    {\thanks{\Ref{FN}{#1}}\newcounter{FNCT@#1}\setcounter{FNCT@#1}{\value{footnote}}}
}

% Create labels for Addresses if the are given in Elsevier format

% Create labels for Footnotes if the are given in Elsevier format

% Part for setting citation format package: natbib

% Part for setting citation format package: biblatex

% Part for indenting CSL refs
% Pandoc citation processing
\newlength{\csllabelwidth}
\setlength{\csllabelwidth}{3em}
\newlength{\cslhangindent}
\setlength{\cslhangindent}{1.5em}
% for Pandoc 2.8 to 2.10.1
\newenvironment{cslreferences}%
  {}%
  {\par}
% For Pandoc 2.11+
\newenvironment{CSLReferences}[2] % #1 hanging-ident, #2 entry spacing
 {% don't indent paragraphs
  \setlength{\parindent}{0pt}
  % turn on hanging indent if param 1 is 1
  \ifodd #1 \everypar{\setlength{\hangindent}{\cslhangindent}}\ignorespaces\fi
  % set entry spacing
  \ifnum #2 > 0
  \setlength{\parskip}{#2\baselineskip}
  \fi
 }%
 {}
\usepackage{calc} % for calculating minipage widths
\newcommand{\CSLBlock}[1]{#1\hfill\break}
\newcommand{\CSLLeftMargin}[1]{\parbox[t]{\csllabelwidth}{#1}}
\newcommand{\CSLRightInline}[1]{\parbox[t]{\linewidth - \csllabelwidth}{#1}\break}
\newcommand{\CSLIndent}[1]{\hspace{\cslhangindent}#1}
% Pandoc header
\usepackage{booktabs}
\usepackage{longtable}
\usepackage{array}
\usepackage{multirow}
\usepackage{wrapfig}
\usepackage{float}
\usepackage{colortbl}
\usepackage{pdflscape}
\usepackage{tabu}
\usepackage{threeparttable}
\usepackage{threeparttablex}
\usepackage[normalem]{ulem}
\usepackage{makecell}
\usepackage{xcolor}

\begin{document}

\title{Supporting Information}

\author{%
%
% Code for old style authors field
%
% Add \and if both authors and author
%
%
% Code for new (elsevier) style author field
%
}

\abstract{}

\date{\today}

\keywords{}

\maketitle



\hypertarget{reminder-knowledge-scales}{%
\section{REMINDER knowledge scales}\label{reminder-knowledge-scales}}

Both the general and the immigration scale derived from the REMINDER
data set (Meltzer et al. 2020) were constructed using \texttt{mirt}
(Chalmers 2012).

\hypertarget{general-knowledge-scale}{%
\subsection{General knowledge scale}\label{general-knowledge-scale}}

The three general items loaded well onto the factor: G1 at 0.51, G2 at
0.894, and G3 at 0.511. The discrimination value for each was acceptable
(in each case greater than 1), and the test information plot suggested
that the test provided most information right below \(\theta\) = 0,
corresponding to mean estimated ability. Unidimensionality was
investigated using parallel analysis by way of the \texttt{psych}
(Revelle 2022) package, which suggested one factor/dimension. Local
independence was investigated using Yen's Q3 (Yen 1993). The largest Q3
value was -0.411. Yen suggests a cut-off value of 0.2, but as pointed
out by Ayala (2009), a Q3 test tends to give inflated negative values
for short tests. Indeed, Yen's own suggestion was in the context of
scales with at least 17 items. For that reason, a value of -0.411 would
seem acceptable, given the short scale. Model fit was evaluated using
empirical plots, which suggested acceptable fit. The measured \(\theta\)
values ranged from -1.358 to 0.729, with a mean of 0.

\hypertarget{immigration-knowledge-scale}{%
\subsection{Immigration knowledge
scale}\label{immigration-knowledge-scale}}

The three immigration items loaded well onto the factor as well: I1 at
0.731, I2 at 0.643, and I4 at 0.464 The discrimination values where
acceptable, with all greater than 1, except for I4 at 0.891. Parallel
analysis suggested one factor/dimension, and the test information plot
showed the scale to discriminate best right below \(\theta\) = 0. Local
independence was investigated using Yen's Q3, with the largest value
being -0.318, which in light of the above observation about short scales
should be considered acceptable. The empirical plots suggested an
acceptable fit. The measured \(\theta\) values ranged from -1.166 to
0.909, with a mean of 0.

\hypertarget{bes-knowledge-scale}{%
\section{BES knowledge scale}\label{bes-knowledge-scale}}

The BES data was taken from Wave 17 of the 2014-2023 British Election
Study Internet Panel (Fieldhouse 2020) (N = 34,366). 0.4\% of the data
was missing across the variables used (education, age, gender, and four
knowledge items). These were imputed using multiple imputation in R's
\texttt{Hmisc} package (Harrell (2022)). The BES knowledge scale, too,
was constructed using \texttt{mirt} (Chalmers 2012). The four items
loaded well onto the factors: BES1 at 0.7, BES2 at 0.852, BES3 at 0.725,
and BES4 at 0.653 The discrimination value for each was acceptable, in
each case greater than 1, and the test information plot suggested that
the test provided most information right below \(\theta\) = 0.
Unidimensionality was investigated using parallel analysis in the
\texttt{psych} (Revelle 2022) package, which suggested one
factor/dimension. Local independence was investigated using Yen's Q3
(Yen 1993). The largest Q3 value was -0.365. For reasons discussed
above, this would seem an acceptable value, given the short scale. Model
fit was evaluated using empirical plots, which suggested acceptable fit.
The measured \(\theta\) values ranged from -1.332 to 1.007, with a mean
of 0.

\hypertarget{anes-knowledge-scale}{%
\section{ANES knowledge scale}\label{anes-knowledge-scale}}

The ANES data was taken from the 2019 Pilot Study (ANES 2019) (N =
3,000). 0.2\% of the data was missing across the variables used
(education, age, gender, and three knowledge items). These were imputed
using multiple imputation in R's \texttt{Hmisc} package (Harrell
(2022)). The knowledge scale was constructed using \texttt{mirt}
(Chalmers 2012). The three items loaded well onto the factors: ANES1 at
0.954, ANES2 at 0.915, and ANES3 at 0.738 The discrimination value for
each was acceptable, in each case greater than 1, and the test
information plot suggested that the test provided most information at
around \(\theta\) = 0. Unidimensionality was investigated using parallel
analysis in the \texttt{psych} (Revelle 2022) package, which suggested
one factor/dimension. Local independence was investigated using Yen's Q3
(Yen 1993). The largest Q3 value is -0.796. For reasons discussed above,
this would seem an acceptable value, given the short scale. Model fit
was evaluated using empirical plots, which suggested acceptable fit. The
measured \(\theta\) values ranged from -1.095 to 0.902, with a mean of
0.

\hypertarget{cces-knowledge-scale}{%
\section{CCES knowledge scale}\label{cces-knowledge-scale}}

The CCES data was taken from the 2020 Cooperative Election Study
(Ansolabehere and Luks 2021) (N = 61,000). 765 of the observations had
missing data in place of the correct answer (e.g., the correct political
affiliation of the relevant state governor), and could as such not be
coded for accuracy or in accuracy. These observations where therefore
removed. An additional 97 observations were missing at least one
response on the knowledge questions. Given the small number of missing
responses here, a decision was taken to simply remove these rather than
impute them. The knowledge scale was constructed using \texttt{mirt}
(Chalmers 2012). The three items loaded well onto the factors: CCES1 at
0.911, CCES2 at 0.925, CCES3 at 0.913, and CCES4 at 0.86. The
discrimination value for each was acceptable, in each case greater than
1, and the test information plot suggested that the test provided most
information at around \(\theta\) = 0. Unidimensionality was investigated
using parallel analysis in the \texttt{psych} (Revelle 2022) package,
which suggested one factor/dimension. Local independence was
investigated using Yen's Q3 (Yen 1993). The largest Q3 value is -0.388.
For reasons discussed above, this would seem an acceptable value, given
the short scale. Model fit was evaluated using empirical plots, which
suggested acceptable fit. The measured \(\theta\) values ranged from
-1.611 to 0.635, with a mean of 0.

\hypertarget{climate-change-knowledge-scale}{%
\section{Climate change knowledge
scale}\label{climate-change-knowledge-scale}}

The data set used to construct the climate change knowledge scale was
derived from the survey ``Public Perceptions of Climate Change across
Four European Countries: United Kingdom, France, Germany and Norway''
(Pidgeon 2016). 0.5\% of the data was missing across the variables used
(education, age, gender, and the three knowledge items). These were
imputed using multiple imputation in R's \texttt{Hmisc} package (Harrell
(2022)). The knowledge scale was constructed using \texttt{mirt}
(Chalmers 2012). The three items loaded well onto the factors: CC1 at
0.938, CC2 at 0.718, and CC3 at 0.437. The discrimination value for each
was acceptable, in each case greater than 1, except for COV3, which had
a discrimination value of 0.827. The test information plot suggested
that the test provided most information at around \(\theta\) = -1.5.
Unidimensionality was investigated using parallel analysis in the
\texttt{psych} (Revelle 2022) package, which suggested one
factor/dimension. Local independence was investigated using Yen's Q3
(Yen 1993). The largest Q3 value was -0.435. For reasons discussed
above, this would seem an acceptable value, given the short scale. Model
fit was evaluated using empirical plots, which suggested acceptable fit.
The measured \(\theta\) values ranged from -1.737 to 0.661, with a mean
of 0.

\hypertarget{covid-knowledge-scale}{%
\section{COVID knowledge scale}\label{covid-knowledge-scale}}

The data set used to construct the COVID knowledge scale was derived
from a separate survey administered by the authors in connection with a
separate project in July 2020. The scale was constructed using
\texttt{mirt} (Chalmers 2012), and the three items loaded well onto the
factors: COV1 at 0.589, COV2 at 0.628, and COV3 at 0.618 The
discrimination value for each was acceptable, in each case greater than
1, except for COV3, which had a discrimination value of 0.827. The test
information plot suggested that the test provided most information at
around \(\theta\) = -2. Unidimensionality was investigated using
parallel analysis in the \texttt{psych} (Revelle 2022) package, which
suggested one factor/dimension. Local independence was investigated
using Yen's Q3 (Yen 1993). The largest Q3 value is -0.192. Model fit was
evaluated using empirical plots, which suggested acceptable fit. The
measured \(\theta\) values ranged from -1.863 to 0.478, with a mean of
0.

\hypertarget{references}{%
\section*{References}\label{references}}
\addcontentsline{toc}{section}{References}

\hypertarget{refs}{}
\begin{CSLReferences}{1}{0}
\leavevmode\vadjust pre{\hypertarget{ref-anes2019}{}}%
ANES. 2019. {``2019 Pilot Study.''} 2019.
\url{https://electionstudies.org/data-center/2019-pilot-study/}.

\leavevmode\vadjust pre{\hypertarget{ref-ansolabehere2021}{}}%
Ansolabehere, Brian F. Schaffner, Stephen, and Sam Luks. 2021.
{``Cooperative Election Study, 2020: Common Content.''} Cambridge, MA:
Harvard University. 2021. \url{http://cces.gov.harvard.edu/}.

\leavevmode\vadjust pre{\hypertarget{ref-deayala2009}{}}%
Ayala, R J de. 2009. \emph{The Theory and Practice of Item Response
Theory}. Methodology in the Social Sciences. New York, NY: Guilford
Publications.

\leavevmode\vadjust pre{\hypertarget{ref-chalmers2012}{}}%
Chalmers, R. Philip. 2012. {``Mirt: A Multidimensional Item Response
Theory Package for the r Environment.''} \emph{Journal of Statistical
Software} 48 (6): 1--29. \url{https://doi.org/10.18637/jss.v048.i06}.

\leavevmode\vadjust pre{\hypertarget{ref-fiedhouse2020}{}}%
Fieldhouse, J. Green, E. 2020. {``British Election Study Internet Panel
Wave 17.''} 2020.

\leavevmode\vadjust pre{\hypertarget{ref-harrell2022}{}}%
Harrell, Frank. 2022. \emph{Hmisc: Harrell Miscellaneous}.
\url{https://CRAN.R-project.org/package=Hmisc}.

\leavevmode\vadjust pre{\hypertarget{ref-reminder2020}{}}%
Meltzer, Christine, Jakob-Moritz Eberl, Nora Theorin, Fabienne Lind,
Tobias Heidenreich, Sebastian Galyga, Hajo G. Boomgaarden, Jesper
Strömbäck, and Christian Schemer. 2020. {``{REMINDER: Online-Panel Study
on Migration and Mobility Attitudes 2017-2018 (SUF edition)}.''} AUSSDA.
\url{https://doi.org/10.11587/LBSMPQ}.

\leavevmode\vadjust pre{\hypertarget{ref-pidgeon2016}{}}%
Pidgeon, N. 2016. {``Public Perceptions of Climate Change Across Four
European Countries: United Kingdom, France, Germany and Norway, 2016.''}
2016.
\url{https://beta.ukdataservice.ac.uk/datacatalogue/doi/?id=8325\#!\#1}.

\leavevmode\vadjust pre{\hypertarget{ref-revelle2022}{}}%
Revelle, William. 2022. \emph{Psych: Procedures for Psychological,
Psychometric, and Personality Research}. Evanston, Illinois:
Northwestern University. \url{https://CRAN.R-project.org/package=psych}.

\leavevmode\vadjust pre{\hypertarget{ref-yen1993}{}}%
Yen, Wendy M. 1993. {``Scaling Performance Assessments: Strategies for
Managing Local Item Dependence.''} \emph{Journal of Educational
Measurement} 30 (3): 187--213.
https://doi.org/\url{https://doi.org/10.1111/j.1745-3984.1993.tb00423.x}.

\end{CSLReferences}






\end{document}
